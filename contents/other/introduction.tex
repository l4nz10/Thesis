%
In ogni grande città di oggi viene offerto un vasto e complesso servizio di rete trasporti pubblici, svolto a garantire un sistema alternativo di trasporto per i cittadini che non possono o non vogliono utilizzare le vetture. Il problema di questi servizi risiede nella loro enorme struttura, la quale certe volte rende il sistema inefficiente portando a lunghe attese dei mezzi di trasporto. Vi è bisogno dunque di un sistema che permetta al cittadino di poter conoscere in anticipo la disponibilità dei mezzi e la loro ubicazione, in modo da ottimizzare al più possibile il tempo a sua disposizione. Nell'era moderna il tempo ormai è la cosa più preziosa a disposizione, e dunque la fornitura di questo sistema assume un ruolo ancora più importante nell'insieme. A maggior ragione, il sistema di monitoraggio deve risultare il più efficiente ed intuitivo possibile, in modo tale che il cittadino possa ottenere ciò che desidera evitando perdite di tempo. L'obiettivo di questa tesi è dunque la realizzazione di un servizio web capace di estrarre i dati necessari da una risorsa web esterna e visualizzarle all'utente in maniera più semplice e diretta. Il lavoro è stato svolto da un team di due persone, dove io mi sono concentrato sulla richiesta dei dati e la loro opportuna visualizzazione mentre il mio collega si è focalizzato sull'apparato di estrazione dei dati.\\

Nel capitolo \ref{chapter:introduzione_al_problema} si descrive il concetto del problema, passando poi ad un'analisi approfondita dell'ecosistema su cui è stata posta l'attenzione ed una descrizione generale delle soluzioni e tecniche adottate per la realizzazione del servizio oggetto di questa tesi.\\
\newpage

Il capitolo \ref{chapter:architettura} definisce la struttura del servizio ed i suoi requisiti architetturali, descrivendo l'architettura server-client ed il concetto di applicazione web moderna. In questa sezione viene specificato il comportamento di una webApp tramite il sistema di richieste AJAX ed il passaggio di informazioni tra server e client attraverso il formato JSON. Il capitolo prosegue descrivendo le specifiche del pattern MVC e le direttive REST, indispensabile per la progettazione di una buona applicazione web.\\

Nel capitolo \ref{chapter:modellazione} viene definito il modello di dominio, fornendo una breve introduzione all'analisi e programmazione orientata agli oggetti ed al linguaggio UML. Segue quindi una precisa descrizione dei significati e funzioni delle componenti di un diagramma del modello di dominio, giungendo dunque alla definizione dei casi d'uso, le classi concettuali di interesse e concludendo con l'impostazione del diagramma del modello di dominio di riferimento per questo progetto.\\

Il capitolo \ref{cha:progettazione} introduce alla progettazione del diagramma delle classi di progetto, proseguendo quanto introdotto nel capitolo precedente e portando ogni aspetto sotto un punto di vista applicativo, definendo le classi che dovranno essere realizzate nel progetto specificando i loro attributi e metodi, inoltre vengono definite tutte le associazioni e le dipendenze necessarie a far sì che le classi possano interagire l'una con l'altra.\\
\newpage

Conclusa la fase di progettazione, nel capitolo \ref{cha:frameworks} si descrivono le scelte di sviluppo, fornendo inizialmente una breve introduzione al linguaggio utilizzato, per passare poi allo studio di tre framework largamente utilizzati per lo sviluppo di applicazioni web: Ember.js, Spine.js e Backbone.js. Per ognuno viene dunque fornita una descrizione dei suoi aspetti principali e dei moduli e dei moduli a disposizione, valutando i vantaggi e gli svantaggi che ogni framework offre. Dopo aver completato la descrizione dei tre framework si apre una sezione sulla valutazione e la scelta del framework che verrà utilizzato per lo sviluppo del software, ponendo le loro caratteristiche a confronto.\\

Nel capitolo \ref{cha:realizzazione} viene descritta la realizzazione e lo sviluppo del servizio, introducento inizialmente gli svantaggi del linguaggio JavaScript e come questi siano stati risolti, descrivendo brevemente il formato common.js e AMD. Proseguendo si passa alla descrizione delle varie componenti del servizio, basandosi sul rispetto del pattern MVC. Vengono descritti i moduli dei modelli di interesse, il modulo di routing, la tecnica con cui i dati vengono richiesti al server e salvati nel client e la struttura per la gestione e la visualizzazione dei dati: le Viste. Viene quindi definito il concetto di template ed esaminate due librerie di templating in modo da farne utilizzo all'interno dell'applicazione. Si conclude il capitolo con le scelte ed i metodi per la visualizzazione dei dati su mappa, la definizione di responsive design ed i suoi pregi.\\

Il capitolo \ref{cha:dimostrazioni} descrive gli esempi d'uso dell'applicazione, mostrando come il servizio appare all'utente ed come questo funzioni. Come ultimo esempio viene mostrato una dimostrazione del design responsive.

La tesi si chiude col capitolo \ref{cha:conclusioni}, in cui vengono descritti gli obiettivi raggiunti e possibili sviluppi futuri per questo servizio.