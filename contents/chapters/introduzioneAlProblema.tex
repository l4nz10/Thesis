Il settore dei trasporti è in continua evoluzione legata all'introduzione di nuovi materiali meno inquinanti, a titoli di viaggio multifunzioni (bus+treno) e a nuove tecniche di localizzazione dei mezzi, che lo riporta di grande attualità.

L’enorme importanza sociale dell’argomento ci impone lo sviluppo di questo settore lasciato per molti anni in balia di se stesso, contribuendo così ad aumentare e portare al collasso il traffico delle nostre città mentre il cittadino è arrivato alla “esasperazione/disperazione”.
Un semplice lavoro stradale, quale una contenuta pavimentazione, la rottura di una strada, in alcuni casi la modifica dei flussi viari, provocano disagi assai maggiori in oggettiva dimensione dell'evento.

La questione principale è il mancato rispetto degli orari di partenza o di transito che sono per l'utenza motivi di insoddisfazione ed incertezza che accrescono il già difficile rapporto delle aziende con la loro clientela oltre alla critica negativa espressa dai cittadini nei confronti dei servizi offerti/organizzati dall’apparato statale/enti autonomi. A questo va aggiunta un’immagine negativa dell’Italia verso paesi stranieri.

In altri termini, in carenza di uno strumento idoneo che permetta di comunicare con tempestività a conducenti, personale di controllo, utenti, le modificazioni intervenute o le correzioni dei servizi tutto il sistema rischia il collasso, così il mezzo di trasporto non trova la possibilità di auspicare un servizio migliore. E' noto come le comunicazioni via etere siano di breve tempo e assolutamente indispensabili proprio in tutti quei casi in cui occorra tempestività e sicurezza.

Gran parte delle aziende di trasporto che operano nelle maggiori città italiane si sono dotate di software vari di gestione centralizzata in grado di rilevare i tempi di percorrenza degli autobus con la possibilità di rendere noti ai cittadini i tempi di attesa in alcune fermate. Questo ha contribuito a risolvere almeno tre questioni: il coordinamento della rete; l'informazione all'utenza; l’acquisizione dei dati di servizio, ai fini di una migliore utilizzazione dei mezzi pubblici e della predisposizione di adeguati piani di trasporto.
Va riconosciuto che, limitare la propria struttura operativa ad un sistema di ricetrasmissione vocale tra centro e personale sul territorio significa limitare di molto le potenzialità e soprattutto, utilizzarla prevalentemente come contingente, ma non certo per risolvere problemi.
Il primo di essi, coordinamento della rete dei servizi, è la diretta conseguenza dell’adozione del sistema di rilevamento della posizione dei mezzi e della loro rappresentazione grafica per segnalare scostamenti o irregolarità.

L'informazione all'utenza è lo strumento ideale per ristabilire quel colloquio, tra coloro che sono in attesa alle fermate oppure a bordo dei mezzi e l'azienda.

L'acquisizione dei dati ``storici" del servizio rappresenta la fonte preziosa a cui attingere al momento di programmare orari ed itinerari dei nuovi programmi di esercizio.

Poter mantenere sotto generale controllo l'intera rete dei servizi viene ad essere molto importante fornirsi di sistemi informatici adeguati ed è dunque essenziale che ciò accada.

L'elemento coerente a tutto ciò deve però essere la possibilità di conoscenza e di intervento in tempo reale; una rete di servizi sottoposta a continuo monitoraggio ed un centro operativo attivo nel coordinamento sono gli elementi tecnici indispensabili.
Se poi, insieme ai dati di posizionamento, il mezzo invia anche i dati relativi all’affollamento dei passeggeri, ai tempi intercorsi per lo spostamento da un punto all'altro del suo itinerario relativi allo stato di efficienza del mezzo, il centro è in grado non solo di verificare la regolarità dei transiti ma, più in generale, lo ``stato" del servizio.

Ci si trova di fronte, inoltre, ed è proprio questo il motivo ispiratore di questo elaborato, ad un'utenza che ``vuole" essere informata con sufficiente anticipo, elemento che può far discendere la propria opzione di trasporto: una linea rispetto ad un'altra tenendo conto del tempo di attesa o, in altri casi, decidere di posticipare l'inizio del trasferimento.

La precisione perciò è, in questi casi, indispensabile. Viviamo in un momento storico in cui poter decidere come impiegare il tempo è particolarmente importante perché da esso dipende la qualità della vita.

Solo un sistema che si basi su dati ``reali" acquisiti direttamente dalla posizione degli autobus sulle rispettive linee può garantirla.
In questo lavoro, viene messo in risalto il servizio che si vuole offrire al cittadino utente, a colui che, poiché in prima linea, soffre dei disservizi e dei ritardi perché, in particolar modo nelle grandi città affida alla funzionalità del servizio di trasporto lo svolgimento della propria giornata. Nella frenesia dei tempi moderni, in cui si è sempre di corsa e con i minuti contati, è particolarmente importante poter scegliere modo e tempo degli spostamenti quotidiani siano essi riferiti all’attività lavorativa, sociale, al tempo libero o al recupero psico-fisico. Quindi si è deciso di mettere a disposizione del cittadino un’applicazione web ad alta tecnologia  ma di uso semplice.  Con un dispositivo abilitato alla navigazione nel web ogni utente potrà visualizzare la localizzazione degli autobus di interesse sull’intero territorio urbano e potrà scegliere quale linea preferire in relazione alla localizzazione dell’autobus. Egli, visualizzando l’intero percorso di tutte le linee, sceglierà i punti di salita e discesa più opportuni. Questa visualizzazione completa della mappa cittadina sul web è di grande supporto a tutti i cittadini, in particolar modo, a coloro che non sono esperti degli spostamenti all’interno della aree cittadine. Offre, quindi, un valore aggiunto a qualsiasi azienda del settore in termini di servizi resi.

E’ d’obbligo chiarire che, nel nostro paese, ogni azienda adotta un software di gestione centralizzata diverso da un’altra perché diverse sono le esigenze, le strutture, e tante altre cose ancora da città a città. Le varie aziende non hanno svolto un lavoro comune per arrivare agli stessi risultati in  termini di servizi offerti al cittadino quindi, in alcune città, è possibile visualizzare gli orari di attesa su qualche fermata delle principali linee urbane, in altre questo risultato è ancora lontano ma si è lavorato su feed-back tra i tempi di percorrenza e gli affollamenti delle linee, in altre si sono ottenute entrambe le soluzioni sopra dette, ed in altre ancora si è lontano da  qualsiasi risultato che possa ottimizzare tempi di percorrenza e servizi di informazioni ai cittadini.

In ogni caso molte aziende di trasporto si sono dotate di software di gestione centralizzata per offrire un servizio  di informazione agli utenti e lavorano assiduamente per migliorare l’organizzazione di detto servizio in un ottica di ottimizzazione costi-benefici ma i software di cui si sono dotate, anche se contengono molteplici indicazioni: percorsi della linea, orari teorici di transiti agli orari feriali e festivi, ore di punta, di calma e serali, rischiano di fallire gli obiettivi perché di difficile lettura ed interpretazione; ciò senza considerare che forniscono elementi ``teorici" spesso vanificati e concorrono ad introdurre ulteriori elementi di incertezza. Essi devono riguardare i tempi di percorrenza reali, il numero dei passeggeri, la movimentazione alle rispettive fermate - saliti e discesi -, la ciclicità della richiesta di servizio e, più in generale, tutto quanto concorre a determinare orari. In questi casi avere a disposizione dati aggiornati sulla realtà delle situazioni non sono irrazionali richieste, ma può risultare assai utile per dimostrare che a fronte di un indubbio aggiornamento corrispondono analoghi benefici collettivi.

Proprio per la molteplicità delle prestazioni che un sistema integrato di gestione, di ricezione, trasmissione, messaggi e informazioni può offrire deve essere realizzato con attenzione.

La velocità dei flussi viari è facilmente deducibile analizzando i tempi di percorrenza dei bus e non è inutile valutare come gli stessi dati, riferiti ovviamente ai soli tempi di scorrimento del traffico, possano essere inviati anche al gestore dello stesso mantenendo sotto controllo la situazione viaria cittadina. Sarà quindi possibile per la stessa entità utilizzarli e, se nel caso, disporre una diversa temporizzazione dei cicli semaforici dando la necessaria priorità, rispetto alle altre componenti del traffico, a favore del trasporto pubblico.
Analogamente gli stessi dati potrebbero essere utilizzati per informare la veicolazione sia essa pubblica che privata.
Quindi dare un servizio di informazione precisa all’utente, oggi, con i potenti mezzi che l’elettronica offre,  è possibile e, soprattutto, si può aggiungere che, il cittadino ha bisogno di sentirsi al centro dell’attenzione, di sentirsi importante, di sentirsi parte integrante delle decisioni.

Un settore di così articolata composizione e difficoltà, non consente di affrontare una gestione centralizzata di un servizio di trasporto, se ad esso non concorrono conoscenze approfondite che difficilmente sono presenti in una singola entità.
Questo lavoro è stato pensato e creato mettendo il cittadino utente al centro dell’interesse. Quello che si raggiunge come scopo finale è la visualizzazione su mappa degli autobus di linea  che ogni utente potrà visualizzare ovunque si trovi utilizzando un dispositivo capace di navigare in rete (smart-phone, tablet, …)

\section{Obiettivo della tesi} % (fold)
\label{sec:section_name}

Come è stato appena introdotto, l'obiettivo di questa tesi è realizzare dunque un'applicazione web che permetta all'utenza di una grande città la consultazione di un servizio di trasporti pubblici, nel particolare la rete autobus urbana.
Per fare ciò vi è bisogno di progettare un servizio web che possa gestire gli accessi dell'utenza ed elaborare le informazioni che vengono richieste, per poi poter visualizzare i risultati di interesse all'utente.

E' stato preso come riferimento quindi il sistema dei trasporti pubblici romani ma, non potendo attingere a dati primitivi come, ad esempio, le coordinate GPS degli autobus in circolazione od una struttura chiara dell'intera ramificazione delle linee di Roma, si è scelto di adottare un sistema di estrazione dei dati direttamente dal sito web che monitora la suddetta rete.
Altro problema già introdotto nella parte iniziale di questo capitolo risiede nella poca chiarezza della visualizzazione dei dati di interesse all'utente. Attualmente il sito web adibito al monitoraggio delle linee autobus gestisce  un basilare sistema di visualizzazione testuale, offrendo all'utente solamente dei tempi di attesa sui prossimi autobus in arrivo in ogni stazione, proprio come accade quando si attende un autobus in una fermata.
Disponendo di un servizio avanzato quale internet perché non fornire invece una visione più globale dell'intera rete dei trasporti? Sarebbe possibile ad esempio poter mostrare all'utente {\itshape tutti} gli autobus in circolazione su una direzione, visualizzando dove essi si trovano in ogni momento.
Su questo principio si basa il fulcro di questo servizio, e al fine di poter realizzare un sistema efficiente, vi è quindi bisogno della risoluzione di due problemi principali:

\begin{enumerate}
    \item estrarre i dati necessari dal sistema dei trasporti pubblici romani
    \item offrire una visualizzazione opportuna dei dati ricavati su una mappa
\end{enumerate}

Per far fronte a queste problematiche viene dunque progettata una struttura server-client in cui il lato client sarà adibito alla richiesta dei dati di interesse e alla loro corretta rappresentazione sia testuale che visuale, mentre il lato server si occuperà di servire le richieste del client ricavando i dati dal sistema web dei trasporti pubblici romani.

Questi aspetti sono stati realizzati attraverso una collaborazione tra due laureandi, lavorando comunemente per la risoluzione delle problematiche di estrazione e di visualizzazione.

Questo elaborato si focalizza sulla gestione lato client dei dati di interesse, descrivendo i metodi di richiesta e di organizzazione dei dati, come essi vengano visualizzati sulla mappa e la struttura dell'interfaccia web.

% section section_name (end)

\newpage
