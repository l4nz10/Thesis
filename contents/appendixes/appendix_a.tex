Di seguito la descrizione delle interfacce API fornite dal server:

\begin{description}
	\item[Risorsa] \textbf{GET /lines} fornisce la lista delle linee a disposizione di cui é possibile eseguire il monitoraggio, non é necessario specificare alcun parametro in questo caso in quanto non vi é alcuna possibilità di filtrare le linee richieste
	\item[Risposta] la risposta sarà restituita dal server in formato JSON e rappresenterà una lista di oggetti linee, di seguito un esempio della rappresentazione.

	\begin{verbatim}
[
  {
    "number": "170"
  },
  {
    "number": "766"
  },
  {
    "number": "23"
  }
]

	\end{verbatim}
\end{description}

\begin{description}
	\item[Risorsa] \textbf{GET /directions} fornisce la lista delle direzioni inerenti alle linee selezionate.
  \item[Parametro] \textbf{lines=<linenumber1>-<linenumber2>-...-<linenumber\#>} è il parametro contenente i codici di linea che viene comunicato al server
  tramite query string.
	\item[Risposta] la risposta sarà restituita dal server in formato JSON e rappresenterà una lista di oggetti direzione, di seguito un esempio della rappresentazione.

	\begin{verbatim}
[
  {
    "id": "1700",
    "name": "170 Direzione Termini (MA-MB-FS)"
  },
  {
    "id": "1701",
    "name": "170 Direzione Agricoltura"
  },
  ...
  { 
    "id": "7660",
    "name": "766 Direz. Millevoi"
  }
]
	\end{verbatim}
\end{description}

\newpage
\begin{description}
	\item[Risorsa] \textbf{GET /stations} fornisce la lista delle fermate disponibili nella direzione selezionata.
  \item[Parametro] \textbf{directions=<directionID1>-<directionID2>-...-<directionID\#>} è il parametro contenente gli identificatori di direzione che viene comunicato al server tramite query string.
	\item[Risposta] la risposta sarà restituita dal server in formato JSON e rappresenterà una lista di oggetti fermate, di seguito un esempio della rappresentazione.

	\begin{verbatim}
[
  {
    "id":"17000",
    "name":"Termini (MA-MB-FS)",
    "iconId":"1",
    "lat":41.901718,
    "lng":12.50009,
    "type":"start"
  },
  {
    "id":"17001",
    "name":"Repubblica (MA)",
    "iconId":"1",
    "lat":41.902557,
    "lng":12.497091,
    "type":"station"
  },
  ...
}
	\end{verbatim}
\end{description}

\newpage
\begin{description}

	\item[Risorsa] \textbf{GET /busses} fornisce la lista degli autobus disponibili sulla direzione selezionata.
  \item[Parametro] \textbf{directions=<directionID1>-<directionID2>-...-<directionID\#>} è il parametro contenente gli identificatori di direzione che viene comunicato al server tramite query string.
	\item[Risposta] la risposta sarà restituita dal server in formato JSON e rappresenterà una lista di oggetti autobus, di seguito un esempio della rappresentazione.

	\begin{verbatim}
[
  {
  "id":"71400",
  "name":"Autobus in: Colombo/Marconi",
  "iconId":1,
  "lat":41.832781,
  "lng":12.470668,
  "type":"bus"
  },
  ...
]
	\end{verbatim}
\end{description}