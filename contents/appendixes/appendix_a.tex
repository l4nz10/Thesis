Di seguito la descrizione delle interfacce API fornite dal server:

\begin{description}
	\item[Resource] \textbf{GET \lines} fornisce la lista delle linee a disposizione di cui é possibile eseguire il monitoraggio, non é necessario specificare alcun parametro in questo caso in quanto non vi é alcuna possibilità di filtrare le linee richieste
	\item[Response] la risposta sarà restituita dal server in formato json e rappresenterà una lista di oggetti linee, di seguito un esempio della rappresentazione.

	\begin{verbatim}
[
  {
    "number": "170"
  },
  {
    "number": "766"
  },
  {
    "number": "23"
  }
]
	\end{verbatim}
\end{description}

\begin{description}
	\item[Resource] \textbf{GET \lines} fornisce la lista delle linee a disposizione di cui é possibile eseguire il monitoraggio, non é necessario specificare alcun parametro in questo caso in quanto non vi é alcuna possibilità di filtrare le linee richieste
	\item[Response] la risposta sarà restituita dal server in formato json e rappresenterà una lista di oggetti linee, di seguito un esempio della rappresentazione.

	\begin{verbatim}
[
  {
    "number": "170"
  },
  {
    "number": "766"
  },
  {
    "number": "23"
  }
]
	\end{verbatim}
\end{description}

\begin{description}
	\item[Resource] \textbf{GET \lines} fornisce la lista delle linee a disposizione di cui é possibile eseguire il monitoraggio, non é necessario specificare alcun parametro in questo caso in quanto non vi é alcuna possibilità di filtrare le linee richieste
	\item[Response] la risposta sarà restituita dal server in formato json e rappresenterà una lista di oggetti linee, di seguito un esempio della rappresentazione.

	\begin{verbatim}
[
  {
    "number": "170"
  },
  {
    "number": "766"
  },
  {
    "number": "23"
  }
]
	\end{verbatim}
\end{description}

\begin{description}
	\item[Resource] \textbf{GET \lines} fornisce la lista delle linee a disposizione di cui é possibile eseguire il monitoraggio, non é necessario specificare alcun parametro in questo caso in quanto non vi é alcuna possibilità di filtrare le linee richieste
	\item[Response] la risposta sarà restituita dal server in formato json e rappresenterà una lista di oggetti linee, di seguito un esempio della rappresentazione.

	\begin{verbatim}
[
  {
    "number": "170"
  },
  {
    "number": "766"
  },
  {
    "number": "23"
  }
]
	\end{verbatim}
\end{description}